\documentclass[11pt]{article}

\usepackage{mathtools}
\usepackage[sc]{mathpazo}
\linespread{1.05} % Palatino needs more leading (space between lines)
\usepackage[T1]{fontenc}
\usepackage{amsmath}
\usepackage{amssymb}
\usepackage{accents}
\usepackage{ntheorem}
\usepackage{latexsym}
\usepackage{xspace}
\usepackage{fancyhdr}
\usepackage{graphicx}
\usepackage{hyperref}
\usepackage[ruled]{algorithm2e}
\usepackage{float}
\usepackage{subfigure}
\usepackage{color}
\usepackage{fullpage}

%%%%%%%% THEOREM TYPE ENVIRONMENTS
\theoremheaderfont{\scshape}
\theorembodyfont{\slshape}
\newtheorem{theorem}{Theorem}
\newtheorem{corollary}[theorem]{Corollary}
\newtheorem{lemma}[theorem]{Lemma}
\newtheorem{proposition}[theorem]{Proposition}
\theoremstyle{plain}
\theorembodyfont{\rmfamily}
\newtheorem{problem}{Problem}
\newtheorem{observation}[theorem]{Observation}
\newtheorem{remark}[theorem]{Remark}
\theorembodyfont{\itshape}
\newtheorem{fact}[theorem]{Fact}
\newtheorem{definition}[theorem]{Definition}
\newtheorem{assumption}[theorem]{Assumption}
\newtheorem{claim}[theorem]{Claim}
\newenvironment{proof}{\noindent {\sc Proof:}}{$\Box$ \medskip}
\newenvironment{proofof}[1]{\noindent {\sc Proof of #1:}}{$\Box$ \medskip}
\theoremstyle{plain}
\theorembodyfont{\normalfont}
\theoremheaderfont{\normalfont\bfseries}
\newtheorem{exercise}{Exercise}

%%%%%%% FORMATTING
\setlength{\topmargin}{-0.5in}
\setlength{\textwidth}{6.5in} % can be up to 6.5
\setlength{\textheight}{9in}
\setlength{\evensidemargin}{-.1in}
\setlength{\oddsidemargin}{-.1in}

\newcounter{lecnum}
\definecolor{hintgray}{gray}{0.40}
\newcommand{\hint}[1]{\textcolor{hintgray}{(Hint:~ #1)}}

%%%%%%%%% MACROS

\newcommand{\set}[1]{\{{#1}\}}
\newcommand{\C}{\ensuremath{\mathbb{C}}}
\newcommand{\R}{\ensuremath{\mathbb{R}}}
\newcommand{\T}{\ensuremath{\mathsf{T}}}
\newcommand{\Z}{\ensuremath{\mathbb{Z}}}
\newcommand{\pr}[2]{\langle{#1, #2}\rangle}
\newcommand{\eps}{\varepsilon}
\newcommand{\vol}{{\rm vol}}
\renewcommand{\epsilon}{\varepsilon}


%%%%%%%% FORMATTING
\newcommand{\exercisedesc}[2]{%
 \newpage %
 \setcounter{lecnum}{#1}%
 \setcounter{page}{1}%

 \pagestyle{fancy}
 \headheight 30pt
 \lhead{\bf{Spring 2019 \\ Geometric Functional Analysis}}
 \chead{\bf{\large{Exercise #1 \\ Discussion on #2}}}
 \rhead{\bf{D. Dadush, J. Briet \\ Mastermath}}
 \renewcommand{\headrulewidth}{1.2pt}
 \setlength{\headsep}{10pt}
}

\newcommand{\exercisesol}[1]{%
 \newpage %
 \setcounter{lecnum}{#1}%
 \setcounter{page}{1}%

 \pagestyle{fancy}
 \headheight 30pt
 \lhead{\bf{Spring 2019 \\ Geometric Functional Analysis}}
 \chead{\bf{\large{Solutions to \\ Exercise #1}}}
 \rhead{\bf{D. Dadush, J. Briet \\ Mastermath}}
 \renewcommand{\headrulewidth}{1.2pt}
 \setlength{\headsep}{10pt}
}

\newcommand{\solution}[1]{
\ifsolutions
\vspace{-3ex}
\noindent\textbf{Solution:} {\itshape #1}
\vspace{3ex}
\fi
}


%switch between show/hide solutions given by \solution
\newif\ifsolutions\solutionsfalse
%\newif\ifsolutions\solutionstrue

\begin{document}

\ifsolutions
\exercisesol{1}
\else
\exercisedesc{1}{02/26/19}
\fi

\begin{exercise}[Exact Volume of $\ell_p$ Balls]~\\
Let $\Gamma(a) = \int_0^\infty t^{a-1} e^{-t} dt$, $a > 0$ denote the Gamma function.
You will prove that
\[
{\rm vol}_n(B_p^n) = (2\Gamma(1/p)/p)^n/\Gamma(n/p+1).
\]
\begin{enumerate}
\item Show that $\int_{\R^n} e^{-\|x\|_p^p} dx = (2\Gamma(1/p)/p)^n$. \\
\hint{Split the integral along each coordinate and apply a change of variables.}
\item Show that $\int_{\R^n} e^{-\|x\|_p^p} dx = {\rm vol}_n(B_p^n) \Gamma(n/p+1)$.
\\
\hint{Use the identity $e^{-\|x\|_p^p} = \int_{\|x\|_p}^\infty p t^{p-1}
e^{-t^p} dt$}
\end{enumerate}
\end{exercise}

\solution{

\begin{enumerate}
\item By splitting up the integral along coordinates, we get that
\begin{equation}
\label{eq:1-1}
\int_{\R^n} e^{-\|x\|_p^p} dx = \int_{\R^n} \prod_{i=1}^n e^{-|x_i|^p} dx
~\underbrace{=}_{\text{Fubini}}~
\prod_{i=1}^n \int_\R e^{-|x_i|^p} dx_i = (\int_{\R} e^{-|t|^p} dt)^n
= (2 \int_0^\infty e^{t^p} dt)^n.
\end{equation}
Applying the change of variable $t \rightarrow t^{1/p}$, we see that
\begin{equation}
\label{eq:1-2}
\int_0^\infty e^{t^p} dt = \frac{1}{p} \int_0^\infty t^{1/p-1} e^t dt =
\frac{1}{p} \Gamma(1/p).
\end{equation}
Combining~\eqref{eq:1-1},~\eqref{eq:1-2}, get the desired expression
\[
\int_{\R^n} e^{-\|x\|_p^p} = (2\Gamma(1/p)/p)^n.
\]
\item Using the identity, we see that
\begin{align}
\int_{\R^n} e^{-\|x\|_p^p} dx 
      &= \int_{\R^n} \int_{\|x\|_p}^\infty p t^{p-1} e^{-t^p} dt dx \nonumber \\
      &= \int_{\R^n} \int_0^\infty 1[\|x\|_p \leq t] p t^{p-1} e^{-t^p} dt dx \nonumber \\ 
      &= \int_0^\infty  p t^{p-1} e^{-t^p} \int_{\R^n} 1[\|x\|_p \leq t]dx dt 
         \quad \left(\text{ by Fubini }\right) \nonumber \\
      &= \int_0^\infty  p t^{p-1} e^{-t^p} \vol_n(tB_p^n) dt \nonumber \\
      &= \int_0^\infty  p t^{p+n-1} e^{-t^p} \vol_n(B_p^n) dt \label{eq:1-3}.
\end{align}
Applying the change of variable $t \rightarrow t^{1/p}$ to the last line, we get
\begin{equation}
\label{eq:1-4}
\int_0^\infty  p t^{p+n-1} e^{-t^p} dt = 
\int_0^\infty  t^{n/p} e^{-t} dt = \Gamma(n/p+1).
\end{equation}
Combining~\eqref{eq:1-3},~\eqref{eq:1-4} yields the result.
\end{enumerate}
}

\begin{exercise}[Duality between $\ell_2$ norm minimization and ellipsoid
maximation]~\\
Let $A \in \R^{m \times n}$ be a non-singular matrix and $b \in \R^n$. 
\begin{enumerate}
\item Show that 
\[
\min \set{\|x\|_2: x \in \R^m, A^\T x = b} \geq \max \set{ y^\T b: y \in \R^n, \|A
y\|_2 \leq 1}.
\]
\hint{Use Cauchy-Schwarz.}
\item Show that the value of both systems is equal to
$\sqrt{b^\T (A^\T A)^{-1} b}$ and hence both have same value.
\hint{Use the fact that $\|A y\|_2 \leq 1$ defines an ellipsoid.}
\item Let $A_1,\dots,A_k$ be matrices where $A_i \in \R^{m_i \times n}$, $i \in
[k]$, let $\lambda_1,\dots,\lambda_k > 0$ and $b \in \R^n$. Assume that
$\sum_{i=1}^k \lambda _i A_i^\T A_i \succ 0$ (positive definite). Use
the above to show that the systems  
\begin{align*}
&1.~ \min \set{\sqrt{\sum_{i=1}^k \lambda_i \|x_i\|^2}: x_i \in \R^{m_i}, i \in [k],
\sum_{i=1}^k \lambda_i A_i^\T x_i = b} \\
&2.~ \max \set{ b^\T y: y \in \R^n, \sum_{i=1}^k \lambda_i \|A_i y\|^2_2 \leq 1}
\end{align*}
are strong duals to each other. That is, show that  any solution to $(1)$ has
value at least that of $(2)$, and the optimal solutions have same value.
\end{enumerate}
\end{exercise}

\solution{
\begin{enumerate}
\item Let $x \in \R^m$ satisfy $A^\T x = b$ and $\|Ay\|_2 \leq 1$. 
Then
\[
\|x\|_2 \geq \|Ay\|_2\|x\|_2 ~\underbrace{\geq}_{\text{Cauchy-Schwarz}}~
(Ay)^\T x = y^\T b.
\]
\item Since $A$ is non-singular, note that $A^\T A$ is invertible.
Furthermore, $((A^\T A)^{-1})^\T = ((A^\T A)^\T)^{-1} = (A^\T A)^{-1}$, so the
inverse is symmetric. Let $y = (A^\T A)^{-1} b / \sqrt{b^\T (A^\T A)^{-1} b}$.
From here, we get that 
\begin{align*}
y^\T b &= \sqrt{b^\T (A^\T A)^{-1} b} , \\
\|Ay\|_2^2 &= y^\T (A^\T A) y  = b^\T (A^\T A)^{-\T} (A^\T A) (A^\T A)^{-1} b /
(b^\T (A^\T A)^{-1} b) = 1 . \\
\end{align*}
Therefore the value of the right hand side program is at least $\sqrt{b^\T (A^\T
A)^{-1} b}$. 

Let $x = A(A^\T A)^{-1} b$. From here, we get that
\begin{align*}
A^\T x &= (A^\T A)(A^\T A)^{-1} b = b , \\
\|x\|_2^2 &= x^\T x =  b^\T (A^\T A)^{-\T} (A^\T A) (A^\T A)^{-1} b = b^\T (A^\T
A)^{-1} b.
\end{align*}
Thus the value of the left hand side program is at most $\sqrt{b^\T (A^\T
A)^{-1} b}$. By Part 1, both programs therefore has the same value, as needed.

\item Let $A \in \R^{m \times n}$, where $m = \sum_{i=1}^k
m_i$, denote the block diagonal matrix 
\[
A = \begin{pmatrix} \sqrt{\lambda_1} A_1 \\ \sqrt{\lambda_2} A_2 \\ \dots
\\ \sqrt{\lambda_k} A_k \end{pmatrix}.
\]
A direct computation reveals that $\|Ay\|_2^2 = \sum_{i=1}^k \lambda_i
\|A_i y\|_2^2$. By Parts $1$ and $2$, we therefore have that
\[
\max \set{y^\T b: \sum_{i=1}^k \lambda_i \|A_iy\|_2^2 \leq 1} = 
\min \set{\|x\|_2: x \in \R^m, A^\T x = b}.
\] 
We now work on the second hand program to get it in the desired form. 
Let us decompose $x = (x_1,\dots,x_k)$, where $x_i \in \R^{m_i}$. From here, we
see that
\begin{align*}
b &= A^\T x = \sum_{i=1}^k \sqrt{\lambda_i} A_i^\T x_i
            = \sum_{i=1}^k \lambda_i A_i^\T (x_i/\sqrt{\lambda_i}), \\
\|x\|^2 &= \sum_{i=1}^k \|x_i\|^2_2 = \sum_{i=1}^m \lambda_i
\|x_i/\sqrt{\lambda_i}\|_2^2 .
\end{align*}
Therefore, letting $z_i = x_i/\sqrt{\lambda_i}$, $\forall i \in [k]$, we see
that
\[
\min \set{\|x\|_2: x \in \R^m, A^\T x = b} = 
\min \set{\sqrt{\sum_{i=1}^k \lambda_i \|z_i\|_2^2}: z_i \in \R^{m_i}, i \in
[k], \sum_{i=1}^k \lambda_i A_i^\T z_i = b},
\] 
as needed.
\end{enumerate}
}

\begin{exercise}[$1/n$-Concavity of Determinant]
Let $A,B \succ 0$ be $n \times n$ positive definite matrices.
\begin{enumerate}
\item Show that $\det(A)^{1/n} = \min \set{{\rm tr}(AX)/n: X \succ 0,
\det(X)=1}$, where equality on the right hand side is uniquely attained at 
$X = A^{-1} \det(A)^{1/n}$. 
\hint{Recall that the trace of a matrix is the sum of the eigenvalues
while the determinant is the product. Compare the two via the arithmetic
mean - geometric mean (AM-GM) inequality.}
\item Conclude that for $\lambda \in (0,1)$, 
$\det(\lambda A)^{1/n} + (1-\lambda) \det(B)^{1/n} \leq \det(\lambda A +
(1-\lambda)B)^{1/n}$ with equality iff $A \in \R_+ B$. 
\end{enumerate}
\end{exercise}

\solution{

\begin{enumerate}
\item Let $\lambda_1,\dots,\lambda_n > 0$ denote the eigen values
of $A$. These are all positive since $A \succ 0$. 

Let $X = A^{-1} \det(A)^{1/n}$. Recall that $\det(A) = \prod_{i=1}^n \lambda_i >
0$ and that the eigenvalues of $A^{-1}$ are $1/\lambda_1,\dots,1/\lambda_n > 0$.
Since $A^{-\T} = (A^{\T})^{-1} = A^{-1}$ is symmetric, $X$ is positive definite.
Second, $\det(X) = \det(A^{-1}) (\det(A)^{1/n})^n = \det(A)^{-1} \det(A) = 1$.
Thus, the value of the program is at most ${\rm tr}(X A)/n = \det(A)^{1/n} {\rm
tr}(I_n)/n = \det(A)^{1/n}$. 

Now take $X \succ 0$ such that $\det(X)=1$. Letting $X^{1/2}$ denote the unique
positive definite square root of $X$, we see that $AX$ is similar to $X^{1/2}
(AX) X^{-1/2} = X^{1/2} A X^{1/2}$. Since the latter is positive definite, $AX$
has positive eigen values $\gamma_1,\dots,\gamma_n > 0$ and is diagonalizable.
Furthermore, since $\det(AX) = \det(A) \det(X) = \det(A)$, we see that
$\prod_{i=1}^n \gamma_i = \det(A)$. From here, we have that
\[
{\rm tr}(AX)/n = \sum_{i=1}^n \gamma_i/n \geq \prod_{i=1}^n \gamma_i^{1/n} = \det(A)^{1/n}
\]
where the first inequality is a consequence of the AM-GM inequality. By the
equality conditions for AM-GM, the above inequality holds at equality iff
$\gamma_1 = \gamma_2 = \cdots = \gamma_n = \det(A)^{1/n}$. Since $AX$ is
diagonalizable, equality thus implies that $AX = \det(A)^{1/n} I_n$, where $I_n$
is the $n \times n$ identity. That is, equality holds iff $X = A^{-1}
\det(A)^{1/n}$ as needed.   

\item From Part 1, we see that
\begin{align*}
\det(\lambda A + (1-\lambda) B)^{1/n}
&= \min_{\det(X)=1,X \succ 0} {\rm tr}((\lambda A + (1-\lambda) B)X)/n \\
&= \min_{\det(X)=1,X \succ 0} \lambda {\rm tr}(A X)/n + (1-\lambda) {\rm tr}(B X)/n \\
&\geq \lambda \min_{\det(X)=1,X \succ 0} {\rm tr}(AX)/n + 
      (1-\lambda) \min_{\det(X)=1,X \succ 0} {\rm tr}(BX)/n \\
&= \lambda \det(A)^{1/n} + (1-\lambda) \det(B)^{1/n}. 
\end{align*}
By part 1, since $\lambda \in (0,1)$, the above holds at equality iff
\[
(\lambda A + (1-\lambda) B)^{-1} \det(\lambda A + (1-\lambda)B)^{1/n} =
A^{-1} \det(A)^{1/n} = B^{-1} \det(B)^{1/n}. 
\]
The above implies that $A = B \det(A)^{1/n}/\det(B)^{1/n} \Rightarrow A \in \R_+
B$. Now if $A \in \R_+ B$, since $A \succ 0$ this implies that $A = \gamma B$,
with $\gamma > 0$. From here, it is direct to check the that equality
conditions are satisfied, thus proving the statement.  
\end{enumerate}
}

\begin{exercise}[Approximation by a Simplex]~\\
Let $K \subset \R^n$ be a convex body. Let $\Delta = {\rm
conv}(p_1,\dots,p_{n+1})$ denote a maximum volume simplex contained in $K$ and
let $c = \sum_{i=1}^{n+1} p_i/(n+1)$ denote its center.
\begin{enumerate}
\item Show that such a simplex exists and give an example body $K$ where it is
not unique.
\item For $i \in [n+1]$, let $\eta_i$ denote a unit normal to the hyperplane
$H_i = {\rm aff.hull}(p_j: j \in [n+1] \setminus \set{i})$, pointing in the
direction of $p_i$, and let $s_i \in \R^n$ satisfy $H_i = \{x \in \R^n:
\pr{\eta_i}{x} = s_i\}$ (note that by assumption $\pr{\eta_i}{p_i} > s_i$).
Prove that
\[
K \subseteq \set{x \in \R^n: |\pr{\eta_i}{x}-s_i| \leq \pr{\eta_i}{p_i}-s_i}.
\]
\hint{Show that if not one can replace $p_i$ to make a simplex of larger
volume.} 
\item Conclude that $\Delta-c \subseteq K-c \subseteq (n+2)(\Delta-c)$.
\end{enumerate}
\end{exercise}

\solution{
\begin{enumerate}
\item If $K = B_2^n$, then if $\Delta$ is a maximum volume simplex then so is $R
\Delta$ where $R$ is any orthogonal transformation since $RB_2^n = B_2^n$ and
$R$ is measure preserving. Since $\Delta$ is clearly a strict subset of $B_2^n$,
the maximizer cannot be unique.

\item Take $x \in K$. By the base times height formula in $\R^n$, we have that
\begin{align*}
\vol_n(p_1,\dots,p_{i-1},x,p_i,\dots,p_n) 
&= \vol_{n-1}(p_1,\dots,p_{i-1},p_{i+1},\dots,p_n) \cdot {\rm dist}(H_i,x)/n \\
&= \vol_{n-1}(p_1,\dots,p_{i-1},p_{i+1},\dots,p_n) \cdot |\pr{\eta_i}{x}-s_i|/n, 
\end{align*}
where ${\rm dist}(H_i,x)$ is the Euclidean distance between the hyperplane $H_i$
and $x$. By the definition of $\Delta$, plugging in $p_i$ for $x$ above
maximizes volume. Given the above formula, this implies that 
\[
\pr{\eta_i}{p_i}-s_i = |\pr{\eta_i}{p_i}-s_i| 
                     \geq |\pr{\eta_i}{x}-s_i| ~ \forall x \in K,
\]
as needed.  
 
\item After shifting $K$ and $\Delta$ by $-c$, we may assume that $0 = c =
\sum_{i=1}^{n+1} p_i/(n+1)$. We now wish to show that $\Delta \subseteq K
\subseteq (n+2)\Delta$. Since the first inclusion is by assumption, we may focus
on proving $K \subseteq (n+2)\Delta$. To begin, we first claim that $\Delta$ can be
expressed in inequality form as follows:
\begin{equation}
\label{eq:4-3}
\Delta = \set{x \in \R^n: \pr{\eta_i}{x} \geq s_i, \forall i \in [n+1]},
\end{equation}
where $s_i$ is as above (after shifting $\Delta$). To see this, let 
\[
A = \begin{pmatrix} p_1 & p_2 & \dots & p_{n+1} \\ 1 & 1 & \dots & 1
\end{pmatrix}.
\]
Note that ${\rm conv}(p_1,\dots,p_{n+1})$ has non-zero volume iff
$p_1,\dots,p_n$ are affinely independent and hence iff $A$ is invertible. From
here, by definition 
\[
x \in {\rm conv}(p_1,\dots,p_{n+1}) \Leftrightarrow A^{-1} \begin{pmatrix} x \\
1 \end{pmatrix} \geq 0. 
\]
To prove the claim it suffices to show that the $i^{th}$ row of $A^{-1}$ is a
positive multiple of $(\eta_i,-s_i)$. Note that this is true iff
$\pr{\eta_i}{p_j}-s_i > 0$ for $i = j$ and $= 0$ otherwise. This last property
follows by construction and hence the claim holds. 

Given the inequality description, to prove that $K \subseteq (n+2)\Delta$ it
suffices to show that
\[
\pr{\eta_i}{x} \geq (n+2)s_i,~ \forall x \in K,~ i \in [n+1].
\]
Take $x \in K$ and $i \in [n+1]$. Since $0 = c \in \Delta$ by assumption, we
have that $s_i \leq 0$ in~\eqref{eq:4-3}. If $\pr{\eta_i}{x} \geq s_i$, the desired
inequality is trivial since $s_i \leq 0$. Thus, we may assume that
$\pr{\eta_i}{x} \leq s_i$. In this situation, by part 2 we have that
\begin{equation}
\label{eq:4-1}
\pr{\eta_i}{p_i}-s_i \geq |\pr{\eta_i}{x}-s_i| = s_i-\pr{\eta_i}{x} .   
\end{equation}
From here, we see that
\begin{align}
0 &= \pr{\eta_i}{c} = \sum_{j \neq i} \pr{\eta_i}{p_j}/(n+1) +
\pr{\eta_i}{p_i}/(n+1) = s_i n/(n+1) + \pr{\eta}{p_i}/(n+1) \nonumber \\
  &\Rightarrow \pr{\eta}{p_i} = - s_i n \label{eq:4-2} .
\end{align}

Combining~\eqref{eq:4-1},~\eqref{eq:4-2}, we have that 
\[
\pr{\eta_i}{x}-s_i \geq s_i-\pr{\eta_i}{p_i} = s_i(n+1)
\Rightarrow \pr{\eta_i}{x} \geq s_i(n+2),  
\]
as needed.
\end{enumerate}
}
\end{document}
