\documentclass[11pt]{article}

\usepackage{mathtools}
\usepackage[sc]{mathpazo}
\linespread{1.05}         % Palatino needs more leading (space between lines)
\usepackage[T1]{fontenc}
\usepackage{amsmath}
\usepackage{amssymb}
\usepackage{accents}
\usepackage{ntheorem}
\usepackage{latexsym}
\usepackage{xspace}
\usepackage{fancyhdr}
\usepackage{graphicx}
\usepackage{hyperref}
\usepackage[ruled]{algorithm2e}
\usepackage{float}
\usepackage{subfigure}
\usepackage{color}
\usepackage{fullpage}

%%%%%%%% THEOREM TYPE ENVIRONMENTS
\theoremheaderfont{\scshape}
\theorembodyfont{\slshape}
\newtheorem{theorem}{Theorem}
\newtheorem{corollary}[theorem]{Corollary}
\newtheorem{lemma}[theorem]{Lemma}
\newtheorem{proposition}[theorem]{Proposition}
\theoremstyle{plain}
\theorembodyfont{\rmfamily}
\newtheorem{problem}{Problem}
\newtheorem{observation}[theorem]{Observation}
\newtheorem{remark}[theorem]{Remark}
\theorembodyfont{\itshape}
\newtheorem{fact}[theorem]{Fact}
\newtheorem{definition}[theorem]{Definition}
\newtheorem{assumption}[theorem]{Assumption}
\newtheorem{claim}[theorem]{Claim}
\newenvironment{proof}{\noindent {\sc Proof:}}{$\Box$ \medskip}
\newenvironment{proofof}[1]{\noindent {\sc Proof of #1:}}{$\Box$ \medskip}
\theoremstyle{plain}
\theorembodyfont{\normalfont}
\theoremheaderfont{\normalfont\bfseries}
\newtheorem{exercise}{Exercise}

%%%%%%% FORMATTING
\setlength{\topmargin}{-0.5in}
\setlength{\textwidth}{6.5in} % can be up to 6.5
\setlength{\textheight}{9in}
\setlength{\evensidemargin}{-.1in}
\setlength{\oddsidemargin}{-.1in}

\newcounter{lecnum}
\definecolor{hintgray}{gray}{0.40}
\newcommand{\hint}[1]{\textcolor{hintgray}{(Hint:~ #1)}}

\newcommand{\lecture}[3]{%
\newpage %
\setcounter{lecnum}{#1}%
\setcounter{page}{1}%

 \thispagestyle{fancy}
 \headheight 45pt
 \lhead{\bf{Mastermath, Spring 2019 \\ Geometric Functional Analysis} \vspace{.5em}\\ \hspace{1em}}
 \chead{{\Large \bf Lecture #1} \vspace{.5em} \\ \hrule height .5mm \hfill \\ {\bf \Large #2}}
 \rhead{\bf{Lecturers: D. Dadush, J. Briet \\ Scribe: #3} \vspace{.5em} \\ \hspace{1em}}
 \renewcommand{\headrulewidth}{1.2pt}
 \hspace{1em} \vspace{6pt}
}

%%%%%%%%% MACROS (ADD YOURS HERE)
\newcommand{\set}[1]{\{{#1}\}}
\newcommand{\T}{\ensuremath{\mathsf{T}}}
\newcommand{\C}{\ensuremath{\mathbb{C}}}
\newcommand{\R}{\ensuremath{\mathbb{R}}}
\newcommand{\Z}{\ensuremath{\mathbb{Z}}}
\newcommand{\N}{\ensuremath{\mathbb{N}}}
\newcommand{\E}{\ensuremath{\mathbb{E}}}
\newcommand{\pr}[2]{\langle{#1, #2}\rangle}
\newcommand{\eps}{\varepsilon}
\newcommand{\vol}{{\rm vol}}
\renewcommand{\epsilon}{\varepsilon}

\begin{document}

\lecture{1}{Geometry and Distances of $\ell_p^n$ Spaces}{D. Dadush}

In these notes, we will give good estimates of the ``size'' of the $\ell_p^n$
unit balls and tight bounds on the distances between $\ell_p^n$ spaces.

\paragraph{\bf Comparing $\ell_p$ norms.} We begin with the basic comparison
inequalities for the $\ell_p$ norms on $\R^n$.

\begin{lemma}
For $1 \leq p < q \leq \infty$, $x \in \R^n$, the following holds:
\[
\|x\|_q \leq \|x\|_p \leq n^{1/p-1/q}\|x\|_q.
\]
Furthermore, the equality cases satisfy:
\begin{enumerate}
\item $\|x\|_q = \|x\|_p$ iff $x$ has at most one non-zero coordinate.
\item $\|x\|_p = n^{1/p-1/q}\|x\|_q$ iff $|x_1|=\cdots=|x_n|$. 
\end{enumerate}
\label{lem:p-q}
\end{lemma}
\begin{proof}
We prove the above for $1 < p < q < \infty$, leaving the remaining cases ($p =
1$ or $q = \infty$) as a simple exercise. For the first inequality, we see that 
\begin{align*}
\|x\|_q &= (\sum_{i=1}^n |x_i|^q)^{1/q} 
        \leq (\max_{i \in [n]} |x_i|^{q-p} \sum_{i=1}^n |x_i|^p)^{1/q} 
        = (\max_{i \in [n]} |x_i|)^{(q-p)/q} \cdot \|x\|_p^{p/q} \\
        &\leq ((\sum_{i \in [n]} |x_i|^p)^{1/p})^{(q-p)/q} \cdot \|x\|_p^{p/q} 
        = \|x\|_p^{(q-p)/q} \cdot \|x\|_p^{p/q} = \|x\|_p .
\end{align*}
For the equality condition here, note that the second inequality can only be
tight if $\max_{i \in [n]} |x_i|^p = \sum_{i \in [n]} |x_i|^p$, which is clearly
iff $x$ at most one non-zero coordinate. Since the first inequality is also
tight if $x$ has at most non-zero coordinate (note we also have tightness if all
the non-zero coordinates have the same absolute value in this setting), this
fully describes the equality condition.

For the second inequality, we apply Holder's inequality with $1/r = p/q$ and
$1/s = (q-p)/q$, to get 
\begin{align*}
\|x\|_p &= (\sum_{i=1}^n |x_i|^p \cdot 1)^{1/p}
\leq ((\sum_{i=1}^n |x_i|^{pr})^{1/r}  (\sum_{i=1}^n 1^{s})^{1/s}))^{1/p}
= (\sum_{i=1}^n |x_i|^q)^{1/q} \cdot n^{(q-p)/(pq)} = \|x\|_q \cdot n^{1/p-1/q}. 
\end{align*}
Recall that equality for Holder's inequality holds above iff for some constant
$c \geq 0$ we have $|x_i|^q = c \cdot 1$, $\forall i \in [n]$. Equivalently iff
$|x_1|=\cdots=|x_n|$ as needed. 
\end{proof}

As a direct corollary to the above, we get the following inclusions and distance
upper bounds:

\begin{corollary} For $1 \leq p \leq q \leq \infty$, the following holds:
\begin{enumerate}
\item $B_p^n \subseteq B_q^n \subseteq n^{1/p-1/q} B_p^n$.
\item $d(\ell_p^n,\ell_q^n) \leq n^{1/q-1/q}$.
\end{enumerate}
\label{cor:p-q}
\end{corollary}
\begin{proof}
For part $1$, note that the first inclusion $B_p^n \subseteq B_q^n
\Leftrightarrow \|x\|_q \leq \|x\|_p,~\forall x \in \R^n$ and the second $B_q^n
\subseteq n^{1/p-1/q} B_p^n \Leftrightarrow \|x\|_p \leq n^{1/p-1/q}
\|x\|_q,~\forall x \in \R^n$. As these inequalities are exactly those
in~\ref{lem:p-q}, the statement follows.

For part $2$, we examine the identity map $I_n$ on $\R^n$ as a map from
$\ell_p^n$ to $\ell_q^n$. To upper bound $d(\ell_p,\ell_q)$ by $n^{1/p-1/q}$, it
suffices to show that $\|I_n\|_{p \rightarrow q} \|I_n\|_{q \rightarrow p} \leq
n^{1/p-1/q}$. From here, we have that bounds $\|I_n\|_{p \rightarrow q} \leq 1$
since $\|x\|_q \leq \|x\|_p$ and $\|I_n\|_{q \rightarrow p} \leq n^{1/p-1/q}$
since $\|x\|_p \leq n^{1/p-1/q} \|x\|_q$, as needed. 
\end{proof}

\paragraph{\bf Volumes of $B_p^n$ balls.} We will now compute useful bounds on
the volumes of the $B_p^n$ balls.  Since for any measurable set $A \subseteq
\R^n$, we have $\vol_n(t A) = t^n \vol_n(A)$ (i.e.~$n$-homogeneity of volume)
for $t \geq 0$, a reasonable measure of the ``size'' of $A$ is its normalized
volume $\vol_n(A)^{1/n}$. Somewhat surprisingly, we will be able to estimate
these normalized volumes of all $B_p^n$ balls up to a constant factor by simply
comparing the extreme cases, the octahedron $B_1^n$ and cube $B_\infty^n =
[-1,1]^n$.

Before starting these computations, we first give the $n$-dimensional analog of
the ``base'' $\times$ ``height'' volume in $\R^2$.

\begin{lemma} Let $A \subseteq \R^{n-1}$ be measurable and $t \geq 0$. Then
\[
\vol_n(\set{(x,s): x \in (s/t) A, s \in [0,t]}) = \frac{t}{n} {\rm
vol}_{n-1}(A).
\]
Furthermore, if $A$ is convex then the convex hull 
\[
{\rm conv}((A,0) \cup \{(0,t)\}) = \set{(x,s): x \in (s/t) A, s \in [0,t]} .
\]
\label{lem:base-height}
\end{lemma}
\begin{proof}
For the first part, by Fubini
\begin{align*}
\vol_n(\set{(x,s): x \in (s/t) A, s \in [0,t]}) 
&= \int_0^t \int_{\R^{n-1}} 1_{x \in (s/t) \in A} dx ds \\
&= \int_0^t \vol_{n-1}((s/t) A) ds
= \vol_{n-1}(A) \int_0^t (s/t)^{n-1} ds \\
&= \vol_{n-1}(A) \cdot t \int_0^1 s^{n-1} ds
= \vol_{n-1}(A) \cdot t/n ,
\end{align*}
as needed. We leave the furthermore as an exercise to the reader.
\end{proof}

We give our estimates for the volume of the $\ell_p^n$ balls below:

\begin{lemma} For $n \geq 1$, the following holds:
\begin{enumerate}
\item $\vol_n(B_\infty^n)^{1/n} = 2$.
\item $\vol_n(B_1^n)^{1/n} = 2/(n!)^{1/n} \leq 2e/n$.
\item For $p \in (1,\infty)$, we have
\[
2 n^{-1/p} = \vol_n(n^{-1/q} B_\infty^n)^{1/n} \leq {\rm
vol}_n(B_p^n)^{1/n} \leq \vol_n(n^{1-1/q} B_1^n) \leq 2e n^{-1/p}.
\] 
\end{enumerate}
\label{lem:vol-p}
\end{lemma}
\begin{proof}
For part 1, we have that $\vol_n(B_\infty^n)^{1/n} = \vol_n([-1,1]^n)^{1/n} =
\vol_1([-1,1]) = 2$. 

For part 2, let $\Delta_n = \set{x \in \R^n: x \geq 0, \sum_{i=1}^n x_i \leq
1}$. By symmetry, 
\begin{align*}
\vol_n(B_1^n) &= \vol_n(\set{x \in \R^n: \sum_{i=1}^n |x_i| \leq 1})
= \vol_n(\cup_{s \in \set{-1,1}^n} \set{(x_1 s_1, \dots, x_n s_n) \in \R^n: x
\geq 0, \sum_{i=1}^n x_i \leq 1}) \\
&= \sum_{s \in \set{-1,1}^n} \vol_n(\set{(x_1 s_1, \dots, x_n s_n) \in \R^n: x \in
\Delta_n})
= 2^n \vol_n(\Delta_n) .
\end{align*}
Above, the second to last equality holds since the intersections have measure zero and
the last equality holds since the linear map $(x_1,\dots,x_n)
\rightarrow (x_1 s_1,\dots,s_n x_n)$ is measure preserving for $s \in \set{-1,1}^n$. 

From here, for $n=1$, note that $\Delta_1 = [0,1]$ and hence ${\rm
vol}_1(\Delta_1) = 1$. For $n \geq 2$, it is direct to verify that $\Delta_n =
{\rm conv}((\Delta_{n-1},0) \cup \set{(0,0)})$. Therefore, by
Lemma~\ref{lem:base-height}, we get that $\vol_n(\Delta_n) =
\vol_{n-1}(\Delta_{n-1})/n$. Thus, by induction, $\vol_n(\Delta_n) = 1/(n!)$.
This yields the exact formula $\vol_n(B_1^n)^{1/n} = 2 \vol_n(\Delta_n)^{1/n} =
2/(n!)^{1/n}$. Using the inequality $n^n/(n!) \leq e^n$ (i.e.~expand the Taylor
series), we conclude that $\vol_n(B_1^n)^{1/n} \leq 2e/n$, as desired. 

For part $3$, it follows directly the parts $1$ and $2$ combined with the
inclusions in Corollary~\ref{cor:p-q} part $1$, namely $n^{-1/p} B_\infty^n
\subseteq B_p^n \subseteq n^{1-1/p} B_1^n$.
\end{proof}

We note that it is perhaps surprising that the most ``obvious'' sandwiching
$B_1^n \subseteq B_p^n \subseteq B_\infty^n$ yields lower and upper normalized
volume estimates that are a $\Theta(n)$ factor off from each other. The fact
that the ``reverse'' inclusions are tighter suggests that most of the volume of
the cube $B_\infty^n$ is close to its vertices while most of the volume of
$B_1^n$ is close to the center of its facets. We note that computing an exact
expression for the volume of the $\ell_p^n$ balls can be done relatively simply
and will be the subject of one of the exercises.

\paragraph{\bf Distances between $\ell_p^n$ spaces.}

We will now show that one can get constant factor tight estimates for the
Banach-Mazur distance between $\ell_p^n$ and $\ell_q^n$ for all ranges of $p$
and $q$. In general, it is extremely difficult to compute good estimates for the
Banach-Mazur distance between normed spaces. In particular, lower bounds must
hold against any possible way of mapping one space bijectively into the other.
The $\ell_p^n$ spaces are indeed one of the very few classes spaces for which
these distances are known and thus give nice examples for how such bounds can be
proved. 

The general formula for the distance between $\ell_p^n$ spaces is given below. 

\begin{theorem}[$\ell_p^n$ distances]
\label{thm:p-q-distance}
For $n \in \N$, $1 \leq p \leq q \leq \infty$, the following holds:
\begin{equation}
\label{eq:p-q-distance}
d(\ell_p,\ell_q) = \begin{cases}  n^{1/p-1/q}:& p \leq q \leq 2 \text{ or }
2 \leq p \leq q \\
\Theta(\max \set{n^{1/p-1/2},n^{1/2-1/q}}):& p \leq 2 \leq q
\end{cases}.
\end{equation}
\end{theorem}

The formulas above are self-dual, i.e.~they are invariant under $(p,q)
\rightarrow (q/(q-1),p/(p-1))$ (note the order switches as we ask for $p \geq q$
for the formulas to be valid), which one would expect since Banach-Mazur
distance is invariant under duality in finite dimensions. The precise distances
are known only for the range $2 \leq p \leq q$ and $p \leq q \leq 2$, where it
will turn out that the identity map is optimal. For the range $p \leq 2 \leq q$,
the formula suggests that $d(\ell_p^n,\ell_q^n) = \Theta(1) \max
\set{d(\ell_p,\ell_{p/p-1}),d(\ell_q^n,\ell_{q/(q-1)})}$, is the maximum
distance between $\ell_p^n,\ell_q^n$ and their respective duals. Thus in this
range, we are in fact measuring the distance to one's dual. In this setting, the
optimal map will no longer be the identity and instead will be a certain
Hadamard like orthogonal transformation.

An important point given by the formulas is that the maximum distance between
any $\ell_p^n$ and $\ell_q^n$ is in fact $\Theta(\sqrt{n})$, which is achieved
only for the settings $(p,q) \in \set{(1,2),(2,\infty),(1,\infty)}$. As we will
see in the next lectures, in the worst-case, any two $n$-dimensional normed
spaces are at distance at most $n$. In fact, we will show that any
$n$-dimensional normed space is at distance at most $\sqrt{n}$ from $\ell_2^n$,
which we see is tight from $\ell_2^n$ vs $\ell_1^n$ or $\ell_\infty^n$. For a
detailed accounting of the subject of Banach-Mazur distances, we encourage the
reader to consult the monograph of Tomczak-Jaegermann~\cite{TJ89}.

Interestingly, all of the above bounds are essentially achieved by
``interpolating'' through the extreme cases $(p,q) \in
\set{(1,2),(2,\infty),(1,\infty)}$. The arguments are mostly elementary with the
exception of proving upper bounds on the distances for the ranges $p \leq 2 \leq
q$, where we will need non-obvious upper bounds on the $p$ to $q$ operator norm
of a Hadamard like transformation. For this purpose, we will rely on the
Riesz-Thorin interpolation theorem, which we state below:

\begin{theorem}[Riesz-Thorin Interpolation] 
\label{thm:riesz-thorin}
Let $T \in \mathcal{L}(\R^m,\R^n)$.  Let $p_1,p_2,q_1,q_2 \in [1,\infty]$,
$\theta \in (0,1)$. Define $p_\theta,q_\theta \in [2,\infty]$ to be 
\begin{align*}
p_\theta &= (\theta/p_1 + (1-\theta)/p_2)^{-1} ,\\
q_\theta &= (\theta/q_1 + (1-\theta)/q_2)^{-1} ,
\end{align*}
the corresponding harmonic averages of $p_1,p_2$ and $q_1,q_2$. Then
\[
\|T\|_{p_\theta \rightarrow q_\theta} \leq \|T\|_{p_1 \rightarrow
q_1}^{\theta}\|T\|_{p_2 \rightarrow q_2}^{1-\theta}.
\]
\end{theorem} 

The above theorem allows us to get upper bounds on operator norms that are ``in
between'' bounds we already know. The proof relies on the concept of complex
interpolation of Banach spaces and Hadamard's three lines lemma, which we may
cover towards the end of the course. We content ourselves for now on using this
theorem as a black-box, as it will allow us to get the full picture for
distances between $\ell_p^n$ spaces.

% Before presenting the formal proofs, we give brief geometric sketch here (the
% formal proofs will use a mixture of geometric and operator theoretic reasoning).
% Recall that from the geometric perspective, the Banach-Mazur distance
% between $\ell_p^n$ and $\ell_q^n$ is the same the distance between the norm
% balls $B_p^n$ and $B_q^n$. That is, we would like to find upper and lower bounds
% on the smallest number $d_{p,q}$ such that there exists a linear transformation
% $T$ such that $T B_p^n \subseteq B_q^n \subseteq d_{p,q} TB_p^n$.

We now prove distance bounds for the base cases $p,q \in \set{1,2,\infty}$.

\begin{lemma}
For $n \in N$, the following holds:
\begin{enumerate}
\item $d(\ell_1^n,\ell_2^n) = d(\ell_2^n,\ell_\infty^n) = \sqrt{n}$.
\item $d(\ell_1^n,\ell_\infty^n) = \Theta(\sqrt{n})$.
\end{enumerate}
\label{lem:base-cases}
\end{lemma}

\begin{proof}
For part $1$, by duality it suffices to prove the bound for $\ell_2^n$ and
$\ell_\infty^n$. By corollary~\ref{cor:p-q}, we have that
$d(\ell_2^n,\ell_\infty^n) \leq \sqrt{n}$. Thus it suffices to prove the lower
bound. Let $T \in \mathcal{L}(\R^n)$ be invertible. By
scaling, we may assume that $\|T\|_{2 \rightarrow \infty} = 1$ and thus must
prove that $\|T^{-1}\|_{\infty \rightarrow 2} \geq \sqrt{n}$. To prove the lower
bound on the operator, for the standard basis $e_1,\dots,e_n \in \R^n$, note
that $1 = \|e_i\|_\infty \leq \|T\|_{2 \rightarrow \infty} \|T^{-1} e_i\|_2 =
\|T^{-1} e_i\|_2$.

Let $\eps_1,\dots,\eps_n$ be uniform $\set{-1,1}$ random variables. From here,
by the parallelogram law, we have that
\[
\E[\|T^{-1}(\sum_{i=1}^n \eps_i e_i)\|_2^2] = \E[\sum_{i,j \in [n]} \eps_i
\eps_j \pr{T^{-1} e_i}{T^{-1} e_j}] = \sum_{i=1}^n \pr{T^{-1} e_i}{T^{-1} e_i} =
\sum_{i=1}^n \|T^{-1} e_i\|_2^2 \geq n.
\]
The second to last equality above follows since $\eps_i^2 = 1$, $~\forall i \in
[n]$, and $\E[\eps_i \eps_j] = \E[\eps_i]\E[\eps_j] = 0$ for $i \neq j$. Given
the above, by averaging, there must exist signs $\eps_1,\dots,\eps_n \in
\set{-1,1}$ such that $\|T^{-1}(\sum_{i=1}^n \eps_i e_i)\|_2 \geq \sqrt{n}$.
Since $\|\sum_{i=1}^n \eps_i e_i\|_\infty = 1$, this proves that
$\|T^{-1}\|_{\infty \rightarrow 2} \geq \sqrt{n}$, as needed.

We now prove part $2$. We begin with the lower bound, which will use volumetric
information. Let $T \in \mathcal{L}(\R^n)$ be invertible.
Again by scaling, we may assume that $\|T\|_{1 \rightarrow \infty} = 1$ and thus
we must show that $\|T^{-1}\|_{\infty \rightarrow 1} = \Omega(\sqrt{n})$. By the
first condition, we see that $\|Te_i\|_2 \leq \sqrt{n} \|T e_i\|_\infty \leq
\sqrt{n} \|T\|_{1 \rightarrow \infty} \|e_i\|_1 \leq \sqrt{n}$. By Hadarmard's
inequality, we in fact have that 
\[
|\det(T)|^{1/n} \leq (\prod_{i=1}^n \|Te_i\|_2)^{1/n} \leq \sqrt{n}
\]
Recalling that $B_\infty^n \subseteq \|T^{-1}\|_{\infty \rightarrow 1} TB_1^n$,
we must have 
\begin{align*}
\vol_n(B_\infty^n)^{1/n} &\leq \vol_n(\|T^{-1}\|_{\infty \rightarrow 1} (TB_1^n))^{1/n}
\Leftrightarrow \\
\frac{\vol_n(B_\infty^n)^{1/n}}{|\det(T)|^{1/n}\vol_n(B_1^n)^{1/n}} &\leq
\|T^{-1}\|_{\infty \rightarrow 1} 
\Rightarrow \\
\sqrt{n}/e &\leq \|T^{-1}\|_{\infty \rightarrow 1},
\end{align*}
as needed.

For the upper bound, we prove it when $n$ is a power of $2$, leaving the general
case as an exercise. For $n$ a power of $2$, we will map $\ell_1^n$ to
$\ell_\infty^n$ using the so-called Hadamard transform $H_n$. For each $i \in
[n]$, let $b(i) \in \set{0,1}^{\log_2 n}$ denote the binary expansion of $i-1$,
i.e.~$i-1 = \sum_{j=0}^{\log_2 n-1} 2^{j} b_{j+1}$. Note that $b:[n] \rightarrow
\set{0,1}^{\log_2 n}$ is clearly bijective when $n$ is a power of $2$. With
this indexing, we may express the coefficient matrix of $H_n$ by $(H_n)_{ij} =
(-1)^{\pr{b(i)}{b(j)}}$, $\forall i,j \in [n]$. It is an exercise to check that
the columns of $H_n$ are orthogonal and all have $\ell_2$ norm $\sqrt{n}$. In
particular, we get that $\|H_n x\|_2 = \sqrt{n} \|x\|_2 \Leftrightarrow
\|H_n^{-1}x\|_2 = \|x\|_2/\sqrt{n}$ for all $x \in \R^n$. 
We now show that $\|H_n\|_{1 \rightarrow \infty} \leq 1$ and
$\|H_n^{-1}\|_{\infty \rightarrow 1} \leq \sqrt{n}$. The first inequality is
direct since the columns of $H_n$ have entries in $\pm 1$. The second inequality
is inequality is derived by comparing to $\ell_2$:
\begin{align*}
\|H_n^{-1}y\|_1 &\leq \sqrt{n}\|H_n^{-1}y\|_2 
                = \|y\|_2 \leq \sqrt{n} \|y\|_\infty, ~\forall y \in \R^n.
\end{align*}
Thus, $\|H_n^{-1}\|_{\infty \rightarrow 1} \leq \sqrt{n}$ as needed. 
\end{proof}

We now use the above to prove Theorem~\ref{thm:p-q-distance}.

\begin{proof}[Proof of Theorem~\ref{thm:p-q-distance}]
We start with the range $p \leq q \leq 2$ or $2 \leq p \leq q$. By duality, we
may restrict to range $2 \leq p \leq q$. Applying Lemma~\ref{lem:base-cases} part 1
and Corollary~\ref{cor:p-q} part 2 together with submultiplicativy of
Banach-Mazur distance, we see that
\[
\sqrt{n} = d(\ell^n_2,\ell^n_\infty) \leq d(\ell^n_2,\ell^n_p)
d(\ell^n_p,\ell^n_q) d(\ell^n_q,\ell^n_\infty) \leq n^{1/2-1/p} n^{1/p-1/q}
n^{1/q} = \sqrt{n}.
\]
Given that the left hand side and right hand side are equal, we must have
equality throughout. In particular, we get $d(\ell^n_p,\ell^n_q) =
n^{1/p-1/q}$. 

We move to the range $p \leq 2 \leq q$. By duality, we may assume that $1/p-1/2
\geq 1/2-1/q$. In contrast to the previous case, we will note be able to use the
estimate $d(\ell_1,\ell_\infty) = \Theta(\sqrt{n})$ in a black-box manner,
however the proof will be almost identical to the case of $\ell_1$ vs
$\ell_\infty$.

We begin with the lower bound. Let $T \in \mathcal{L}(\R^n)$ be invertible and
assume that $\|T\|_{p \rightarrow q} = 1$. Let $d = \|T^{-1}\|_{q \rightarrow
p}$. As before, we must prove that $d \geq \Omega(n^{1/p-1/2})$. Note that for
$i \in [n]$,
\[
\|Te_i\|_2 \leq n^{1/2-1/q}\|Te_i\|_q 
           \leq n^{1/2-1/q}\|T\|_{p \rightarrow q} \|e_i\|_q 
           = n^{1/2-1/q},
\]
where the last equality holds by assumption on $T$. Therefore, by Hadamard's
inequality $|\det(T)|^{1/n} \leq n^{1/p-1/2}$. Since $B_q^n \subseteq d TB_p^n$,
using the volume bounds in Lemma~\ref{lem:vol-p}, we get as before that
\[
d \geq \frac{\vol_n(B_q^n)^{1/n}}{|\det(T)|^{1/n} \vol_n(B_p^n)^{1/n}}  
  \geq n^{1/p-1/2}/e,
\]
as needed.

For the upper bound, we assume as before that $n$ is a power of $2$, leaving the
general case as an exercise. We use the Hadamard transformation $H_n$ as before.
Recall that $\|H\|_{2 \rightarrow 2} = \sqrt{n}$ and $\|H\|_{1 \rightarrow
\infty} = 1$. Let $\theta = 2/q$ where $p_\theta = (\theta \cdot 1/2 +
(1-\theta) \cdot 1)^{-1} = q/(q-1)$ and $q_\theta = (\theta \cdot 1/2 +
(1-\theta) \cdot 1/\infty) = q$. Note that $1/p_\theta + 1/q = 1$, so these form
a dual pair. Furthermore, our assumption that $1/p + 1/q \geq 1 \Rightarrow
p_\theta = q/(q-1) \geq p$. Since $\|x\|_{p_\theta} \leq \|x\|_p$, $\forall
x \in \R^n$, we see that 
\[
\|H_n\|_{p \rightarrow q} 
\leq \|H_n\|_{p_\theta \rightarrow q} = \|H_n\|_{p_\theta \rightarrow q_\theta}.
\]
From here, applying Riesz-Thorin interpolation (Theorem~\ref{thm:riesz-thorin}),
we get that 
\[
\|H_n\|_{p_\theta \rightarrow q_\theta} \leq \|H_n\|_{2 \rightarrow 2}^{\theta}
\|H_n\|_{1 \rightarrow \infty}^{1-\theta} = n^{1/q}.
\]
Thus $\|H_n\|_{p \rightarrow q} \leq n^{1/q}$. For the bound on $\|H_n^{-1}\|_{q
\rightarrow p}$, we again compare directly to $\ell_2$, which gives
\[
\|H_n^{-1} y\|_p \leq n^{1/p-1/2} \|H_n^{-1} y\|_2 = n^{1/p-1} \|y\|_2
\leq n^{1/p-1} (n^{1/2-1/q} \|y\|_q) = n^{1/p-1/2-1/q} \|y\|_q, \forall y \in
\R^n.
\]
Thus, $\|H^{-1}\|_{q \rightarrow p} \leq n^{1/p-1/2-1/q}$.  This yields the
final distortion bound 
\[
\|H_n\|_{p \rightarrow q} \|H_n^{-1}\|_{q \rightarrow p} \leq n^{1/q} \cdot
n^{1/p-1/2-1/q} = n^{1/p-1/2},
\]
as needed.
\end{proof}

\bibliographystyle{alpha}
\bibliography{references}

\end{document}
