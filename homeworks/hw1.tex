\documentclass[11pt]{article}

\usepackage{mathtools}
\usepackage[sc]{mathpazo}
\linespread{1.05} % Palatino needs more leading (space between lines)
\usepackage[T1]{fontenc}
\usepackage{amsmath}
\usepackage{amssymb}
\usepackage{accents}
\usepackage{ntheorem}
\usepackage{latexsym}
\usepackage{xspace}
\usepackage{fancyhdr}
\usepackage{graphicx}
\usepackage{hyperref}
\usepackage[ruled]{algorithm2e}
\usepackage{float}
\usepackage{subfigure}
\usepackage{color}
\usepackage{fullpage}

%%%%%%%% THEOREM TYPE ENVIRONMENTS
\theoremheaderfont{\scshape}
\theorembodyfont{\slshape}
\newtheorem{theorem}{Theorem}
\newtheorem{corollary}[theorem]{Corollary}
\newtheorem{lemma}[theorem]{Lemma}
\newtheorem{proposition}[theorem]{Proposition}
\theoremstyle{plain}
\theorembodyfont{\rmfamily}
\newtheorem{problem}{Problem}
\newtheorem{observation}[theorem]{Observation}
\newtheorem{remark}[theorem]{Remark}
\theorembodyfont{\itshape}
\newtheorem{fact}[theorem]{Fact}
\newtheorem{definition}[theorem]{Definition}
\newtheorem{assumption}[theorem]{Assumption}
\newtheorem{claim}[theorem]{Claim}
\newenvironment{proof}{\noindent {\sc Proof:}}{$\Box$ \medskip}
\newenvironment{proofof}[1]{\noindent {\sc Proof of #1:}}{$\Box$ \medskip}
\theoremstyle{plain}
\theorembodyfont{\normalfont}
\theoremheaderfont{\normalfont\bfseries}
\newtheorem{exercise}{Exercise}

%%%%%%% FORMATTING
\setlength{\topmargin}{-0.5in}
\setlength{\textwidth}{6.5in} % can be up to 6.5
\setlength{\textheight}{9in}
\setlength{\evensidemargin}{-.1in}
\setlength{\oddsidemargin}{-.1in}

\newcounter{lecnum}
\definecolor{hintgray}{gray}{0.40}
\newcommand{\hint}[1]{\textcolor{hintgray}{(Hint:~ #1)}}

%%%%%%%%% MACROS

\newcommand{\set}[1]{\{{#1}\}}
\newcommand{\C}{\ensuremath{\mathbb{C}}}
\newcommand{\R}{\ensuremath{\mathbb{R}}}
\newcommand{\T}{\ensuremath{\mathsf{T}}}
\newcommand{\Z}{\ensuremath{\mathbb{Z}}}
\newcommand{\pr}[2]{\langle{#1, #2}\rangle}
\newcommand{\eps}{\varepsilon}
\newcommand{\vol}{{\rm vol}}
\renewcommand{\epsilon}{\varepsilon}


%%%%%%%% FORMATTING
\newcommand{\homeworkdesc}[2]{%
 \newpage %
 \setcounter{lecnum}{#1}%
 \setcounter{page}{1}%

 \pagestyle{fancy}
 \headheight 30pt
 \lhead{\bf{Spring 2019 \\ Geometric Functional Analysis}}
 \chead{\bf{\large{Homework #1 \\ Due #2}}}
 \rhead{\bf{D. Dadush, J. Briet \\ Mastermath}}
 \renewcommand{\headrulewidth}{1.2pt}
 \setlength{\headsep}{10pt}
}

\newcommand{\homeworksol}[1]{%
 \newpage %
 \setcounter{lecnum}{#1}%
 \setcounter{page}{1}%

 \pagestyle{fancy}
 \headheight 30pt
 \lhead{\bf{Spring 2019 \\ Geometric Functional Analysis}}
 \chead{\bf{\large{Solutions to \\ Homework #1}}}
 \rhead{\bf{D. Dadush, J. Briet \\ Mastermath}}
 \renewcommand{\headrulewidth}{1.2pt}
 \setlength{\headsep}{10pt}
}

\newcommand{\solution}[1]{
\ifsolutions
\vspace{-3ex}
\noindent\textbf{Solution:} {\itshape #1}
\vspace{3ex}
\fi
}


%switch between show/hide solutions given by \solution
\newif\ifsolutions\solutionsfalse
%\newif\ifsolutions\solutionstrue

\begin{document}

\ifsolutions
\homeworksol{1}
\else
\homeworkdesc{1}{7/05/18}
\fi

\begin{exercise}[Distance between $\ell_1$ and $\ell_\infty$]~\\
Construct a linear map $T: \ell_1^n \rightarrow \ell_\infty^n$ satisfying
$\|T\|_{\rm op}\|T^{-1}\|_{\rm op} = O(\sqrt{n})$. The goal of this exercise is
to extend the Hadamard based construction in Lecture $1$ from $n$ a power of $2$
to all $n$.  
\hint{Use the binary expansion of $n$.}
\end{exercise}

\begin{exercise}[Little Grothendieck Inequality]

\end{exercise}

\begin{exercise}[Covering with an Asymmetric Convex Body]~\\
Let $B \subseteq \R^n$ be a convex body. In class, we gave volumetric estimates
on covering numbers when $B$ is symmetric, which you will extend to the
asymmetric case here.

\begin{enumerate}
\item Show that $\E[\vol_n(B \cap (2x-B))/\vol_n(B)] = 1/2^n$, where $x$ is
sampled uniformly from $B$. \\
\hint{Show that the above expectation can be expressed as $\vol_n(B)^{-2} \int_B
\int_B 1[y \in (2x-B)] dy dx$ and exchange the order of integration.} 
\item Use the above to show that there exists a symmetric convex body $K
\subseteq \R^n$ and a shift $t \in \R^n$ such that $K+t \subseteq B$ and
$\vol_n(K) \geq 2^{-n} \vol_n(B)$. \\
\hint{Note that $2x-B$ is the reflection of $B$ about $x$.}
\item For $K$ as above and $A \subseteq \R^n$, show that
\[
N(A,B) \leq N(A,K) \leq 4^n \vol_n(A+B/2)/\vol_n(B) \text{ .}
\]  
In particular, show that $N(B,K) \leq 6^n$ and $N(B-B,B) \leq 30^n$. \\ 
\hint{Use the packing argument with $B$ replaced by $K$ and compare the bounds.}
\end{enumerate}
\end{exercise}

\begin{exercise}[Low Rank Approximation of the Identity]
Let $I_n \in \R^{n \times n}$ denote the $n \times n$ identity matrix. For $\eps
\in (0,1)$, show that there exists a positive semidefinte matrix $\tilde{I}_n
\in \R^{n \times n}$ such that $|(I_n-\tilde{I}_n)_{ij}| \leq \eps$, $\forall
i,j \in [n]$ and ${\rm rank}(\tilde{I}_n) = O(\log n / \eps^2)$. \\
\hint{Apply the Johnson-Lindenstrauss lemma to the canonical rank factorization
of $I_n$.}
\end{exercise}

\end{document}
